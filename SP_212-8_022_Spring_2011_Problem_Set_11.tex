\documentclass[makesolutionspdf]{esg8022pset}
\begin{preamble}
\usepackage{amsmath}
\usepackage{amssymb}
\usepackage{enumerate}
\usepackage{graphicx}
\usepackage{hyperref}
\usepackage{mathtools}
\usepackage[per-mode=symbol]{siunitx} %If this line is giving you trouble, try replacing per-mode with per
%use inter-unit-separator={}\cdot{} ?
\providecommand{\uvec}[1]{{\hat{\bf{#1}}}}
\usepackage{pgf,tikz}
\usetikzlibrary{arrows}
\usepackage{wasysym}
\usepackage{subfig}
\makeatletter
\newcommand{\interitemtext}[1]{%
  \begin{list}{}
   {\itemindent=0mm\labelsep=0mm
   \labelwidth=0mm\leftmargin=0mm
   \addtolength{\leftmargin}{-\@totalleftmargin}}
    \item #1
  \end{list}
}
\makeatother
\renewcommand{\d}{\,d}
\providecommand{\norm}[1]{\lVert#1\rVert}

\newcommand{\Kgrad}{\left(\hat{x} \frac{\partial}{\partial x} + \hat{y} \frac{\partial}{\partial y} + \hat{z} \frac{\partial}{\partial z}\right)}
\newcommand{\Kdiv}[6]{{#4}\left(\frac{\partial {#1}}{\partial x} {#5} \frac{\partial {#2}}{\partial y} {#6}\frac{\partial #3}{\partial z} \right)}
\newcommand{\KKdiv}[6]{{#4}\left(\frac{\partial}{\partial x}{#1} {#5} \frac{\partial}{\partial y}{#2} {#6}\frac{\partial}{\partial z}{#3} \right)}
\newcommand{\dx}{\frac{\partial}{\partial x}}
\newcommand{\dy}{\frac{\partial}{\partial y}}
\newcommand{\dz}{\frac{\partial}{\partial z}}
\newcommand{\dtheta}{\frac{\partial}{\partial \theta}}
\newcommand{\dr}{\frac{\partial}{\partial r}}

\AtBeginDocument{%
  % Appologies to any future editor on the inconsistencies in TeX code and the unnecessary braces.  I'm aggregating previously typeset problems, and didn't think it worth my time to improve the quality of TeX code in ways that won't make any difference to the typeset material. -Jason Gross (jgross@mit.edu)
}%
\end{preamble}

\classname{Physics 8.022}
\semester{Spring 2011}
\problemsetnumber{10} %Put the problem set number here
\duedate{Wednesday, May 4th 10 am IN CLASS}
%\readingassignment{Kleppner and Kolenkow, \emph{An Introduction to Mechanics}, Chapters Seven and Eight}
\problemsettitle{Maxwell's equations, waves}

\begin{document}

\begin{problem}{Purcell 9.1}

\begin{figure}[H]
    \centering
    \includegraphics[width = 15cm]{pu901}
    \caption{Purcell 9.1}
  \end{figure}
  \end{problem}
\begin{solution}
  \begin{figure}[H]
    \centering
    \includegraphics[width = 15cm]{solpu901}
    \caption{Solution Purcell 9.1}
  \end{figure}
\end{solution}


\begin{problem}{Purcell 9.5a}
 \begin{figure}[H]
    \centering
    \includegraphics[width = 15cm]{pu905}
    \caption{Purcell 9.5}
  \end{figure}
\end{problem}
\begin{solution}
 \begin{figure}[H]
    \centering
    \includegraphics[width = 15cm]{solpu905a}
    \caption{Solution Purcell 9.5a}
  \end{figure}
\end{solution}


\begin{problem}{Discovery of magnetic charge}
You discover magnetic
charge.  The units of magnetic charge density,
$\mu$, are chosen such that $\vec\nabla\cdot\vec B = 4\pi\mu$.

\par\noindent (a) When this magnetic charge is in motion,
there is a ``magnetic current density'' $\vec L = \mu \vec v$.  In
analogy to electric charge density and electric current densities,
write down the equation of continuity for magnetic charge.

\par\noindent (b) What do Maxwell's equations become with this
new charge?
\end{problem}

\begin{solution}
(a) \begin{equation}
\frac{\partial \mu}{\partial t}+\vec{\nabla}\cdot\vec{L}=0.
\end{equation}
(b) Take the divergence of the $\vec\nabla\times\vec E$ equation; you'll
find that the new equation of magnetic charge continuity is violated.
To fix it, you must add a term that is proportional to $\vec L$ to
Faraday's law.  The resulting Maxwell equations are:

\begin{eqnarray}
\vec{\nabla}\times\vec{E} &=& -\frac{1}{c}\frac{\partial
\vec{B}}{\partial t}-\frac{4\pi}{c}\vec{L}\\
\vec{\nabla}\times\vec{B} &=& \frac{1}{c}\frac{\partial
\vec{E}}{\partial t}+\frac{4\pi}{c}\vec{J}\\
\vec{\nabla}\cdot\vec{E} &=& 4\pi\rho\\
\vec{\nabla}\cdot\vec{B} &=& 4\pi\mu.
\end{eqnarray}




\end{solution}

\begin{problem}{ Magnetic field of a moving charge}
A charge $q$ moving
along the $x$-axis at constant speed $v \ll c$.  When it is at $x =
-d$, what is the magnetic field at $(x,y,z) = (0,r,0)$?

\par\noindent (a)  Solve this first using Biot-Savart.  (Hint:
the current from the moving charge isn't particularly well defined.
However, B-S only needs the combination $I dl = (dq/dt) dl = dq
(dl/dt) \simeq q_{\rm pt\ charge}(dl/dt)$.  Sloppy physicist calculus
in action!)

\par\noindent (b)  Now solve this using displacement current.
Look at a circle of radius $r$ centered at the origin and passing
through the point $(0,r,0)$.  By symmetry, $\vec B$ will be constant
on this circle and oriented in the tangential direction.  Find a
surface which has this circle as a boundary and for which $\int \vec
E\cdot d\vec a$ is simple.  Evaluate this flux, apply the
``generalized'' form of Ampere's law (integral formulation) and you're
there.

\par\noindent Note, there's a third way: Lorentz transform from the
rest frame electric field (which you used on a previous pset).  All
three answers should agree, at least in the limit $v \ll c$.

\end{problem}
\begin{solution}
a) Solve this first using Biot-Savart.\\

For low speed $v\ll c$, we can ignore the relativistic effects.  To
apply B-S law, we simply treat the moving charge at $x=-d$ as an
element current at the same location and $I d\vec{l}=qv\hat{x}$.  Then
we have 
\begin{equation}
\vec{B}= \frac{qv\hat{x}\times\hat{r}_1}{cr_1^2},
\end{equation}
where $\vec{r}_1$ is the position vector from the moving charge to the
test point $(0,r,0)$.  Using $r_1 = \sqrt{r^2 + d^2}$ and $\hat
x\times\hat r_1 = \hat z\sin\theta_1 = \hat z r/r_1$, we find
\begin{equation}
\vec{B}= \hat{z}\frac{qvr}{c(r^2+d^2)^{3/2}}.
\end{equation}

(b) Now solve this using displacement current.\\

 \begin{figure}[H]
    \centering
    \includegraphics[width = 15cm]{m6}
    \caption{Calculation of magnetic field of a moving charge by
``generalized'' Ampere's law.}
  \end{figure}

Consider a surface $\Sigma$ whose boundary is the circle $C$ centered
at the origin and in Y-Z plane, and all points on $\Sigma$ have the
same distance $r_1$ to the moving charge at $(-d,0,0)$.  Apply Stoke's
Theorem,
\begin{eqnarray}
\int_{\Sigma} \vec{\nabla}\times\vec{B}\cdot d\vec{a} &=& \int_C
\vec{B}\cdot d\vec{l}\\
\textrm{while} \qquad\qquad \int_{\Sigma} \vec{\nabla}\times\vec{B}\cdot
d\vec{a} &=& \frac{1}{c}\int_{\Sigma} \frac{\partial \vec{E}}{\partial
t}\cdot d\vec{a}\nonumber\\
&=& \frac{1}{c}\frac{\partial}{\partial t}\int_{\Sigma} \vec{E}\cdot
d\vec{a}\\
\textrm{so}\qquad \int_C \vec{B}\cdot d\vec{l} &=& 
\frac{1}{c}\frac{\partial}{\partial t}\int_{\Sigma} \vec{E}\cdot
d\vec{a}.
\end{eqnarray}

On the surface $\Sigma$, $E=q/r_1^2$ in radial direction (seen from
the moving charge).  So
\begin{eqnarray}
\int \vec{E}\cdot d\vec{a} &=& \frac{q}{r_1^2}\int da\nonumber\\
&=& \frac{q}{r_1^2}r_1^2\int_{0}^{\theta_1}\sin\theta d\theta
\int_0^{2\pi}d\phi\nonumber\\
&=& 2\pi q\left[\cos(0) - \cos(\theta_1)\right]\nonumber\\
&=& 2\pi q(1-\frac{d}{\sqrt{d^2+r^2}}).
\end{eqnarray}

Then, since $d(-d)/dt=v$,
\begin{eqnarray}
\frac{1}{c}\frac{\partial}{\partial t}\int_{\Sigma} \vec{E}\cdot
d\vec{a} &=& \frac{2\pi qv}{c}\frac{r^2}{(d^2+r^2)^{3/2}}\\
\int_C \vec{B}\cdot d\vec{l} &=& 2\pi rB\\
B&=& \frac{qvr}{c(r^2+d^2)^{3/2}}.
\end{eqnarray}


\end{solution}


\begin{problem}{General questions}
\begin{figure}[H]
    \centering
    \includegraphics[width = 15cm]{max_gen}
    \caption{Maxwell's equations}
  \end{figure}
\end{problem}

\begin{problem}{EM waves}
\begin{figure}[H]
    \centering
    \includegraphics[width = 15cm]{waves1}
    \caption{Waves}
  \end{figure}
\end{problem}


\begin{problem}{Loop antenna}
\begin{figure}[H]
    \centering
    \includegraphics[width = 15cm]{loopantenna}
    \caption{Antenna}
  \end{figure}
\end{problem}


\begin{problem}{The Director's Challenge --- Extra credit!!!}
  Formulate an interesting problem that relates a topic from 8.022 to your
  intended major or any other topic about which you are passionate.  Give references
  to help future students to understand the context.  Try to give a solution.
  Any method --- theoretical, analytical, numerical, experimental --- is acceptable.
  If you can't give a full solution, outline partial solutions. Enjoy!
\end{problem}


\begin{problem}{Magnetic monopole:experiments}

 \begin{figure}[H]
    \centering
    \includegraphics[width = 15cm]{monopoles}
    \caption{Magnetic monopole}
  \end{figure}

\end{problem}

\begin{solution}

 \begin{figure}[H]
    \centering
    \includegraphics[width = 15cm]{mono_sol_a}
    \caption{Magnetic monopole}
\end{figure}

 \begin{figure}[H]
    \centering
    \includegraphics[width = 15cm]{mono_sol_b}
    \caption{Magnetic monopole}
\end{figure}

 \begin{figure}[H]
    \centering
    \includegraphics[width = 15cm]{monopole_sol_c}
    \caption{Magnetic monopole}
   \end{figure} 
    
     \begin{figure}[H]
    \centering
    \includegraphics[width = 15cm]{monopole_sol_d}
    \caption{Magnetic monopole}
\end{figure}

\end{solution}





\end{document}
